\documentclass{article}

% encoding package
\usepackage[utf8]{inputenc}

% begining of the document
\begin{document}
	% this is a comment.
	Comments are 100\% cool.
\end{document}
